\section{Постановка задачи}
\label{sec:Chapter1} \index{Chapter1}
Здесь надо максимально формально описать суть задачи, которую потребуется решить, так, чтобы можно было потом понять, в какой степени полученное в результате работы решение ей соответствует. Текст главы должен быть написан в стиле технического задания, т.е. содержать как описание задачи, так и некоторый набор требований к решению

Данная задача решается в два этапа. Изначально необходимо обработать имеющуюся фотографию и найти ключевые точки, а потом уже классифицировать позу человека, исходя из информации об этих точках. Тут задерживаться не стоит и можно переходить к описанию каждой задачи по отдельности.

\subsection{Задача распознавания ключевых точек на теле человека}

В данном подразделе мы рассмотрим постановку задачи распознавания ключевых точек на теле человека, метрики для оценки хорошести работы модели и возможные улучшения с течением времени.

Начнем с того, сколько точек мы должны распознать: 17,25,33. Именно такие варианты сейчас есть на рынке. Возможно и более большее количество точек, если объединять их с моделями распознавания рук и очертаний лица. Но это уже другая история.

\subsection{Задача классификации движений/позы человека}
\newpage