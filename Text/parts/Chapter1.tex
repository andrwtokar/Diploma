\section{Постановка задачи}
\label{sec:Chapter1} \index{Chapter1}
Здесь надо максимально формально описать суть задачи, которую потребуется решить, так, чтобы можно было потом понять, в какой степени полученное в результате работы решение ей соответствует. Текст главы должен быть написан в стиле технического задания, т.е. содержать как описание задачи, так и некоторый набор требований к решению

Как уже было сказано в главе \ref{sec:Chapter0}, будет произведена классификация движений человека на изображении. Из изображения надо получить данные о принимаемой субъектом позе и классифицировать её на род деятельности человека. Получается мы решаем две задачи: предобработки данных, то есть извлечение расположения ключевых точек на теле человека, и их последующая категоризация. Рассмотрим их по отдельности.


\subsection{Задача распознавания ключевых точек на теле человека}
\label{subsec:Theory of keypoint detection}

Первоначально необходимо понять каким образом можно распознать позу человека, чтобы в дальнейшем взять оттуда информацию для классификации. Человек смотрит на другого человека и анализирует его позицию исходя из данных о его расположении частей тела анализируемого. Получается нам необходимо найти части тела человека, каждая из которых ограничена какими-либо суставами. Последние можно и взять за ключевые точки, которые будут распознаваться моделью. Если соединить выходные данные, то получим своеобразный "скелет"{} человека.

Необходимо определиться сколько точек на теле человека необходимо различать. На текущий момент стандартом является топология СОСО (см. \autoref{fig:COCO_topology}), которая включает в себя 17 ориентиров на теле человека \cite{COCO_topology, COCO_dataset}. Данная топология не учитывает расположение ступней и кистей рук, а также рассматривает всего 5 точек на лице человека: нос, два глаза и два уха. Но стандартом многие исследователи не ограничиваются и добавляют дополнительные точки. Приведу два примера:

\begin{enumerate} 
  \item Топология от BlazePose (см. \autoref{fig:BlazePose_topology})\\
  Включает в себя 33 точки расположенные на теле человека. Данная топология представляет собой объединение COCO, BlazeFace \cite{BlazeFace} и BlazePalm \cite{Hands}. В итоге мы получаем дополнительную информацию о направлении стоп и кистей, а также больше понимаем насчет точек на лице. Данная модель расположения точек используется в одноименной модели (BlazePose \cite{BlazePose}) и ориентирована на использование в фитнес приложениях. Также у данной компании есть более развитая модель, которая определяет положение всех пальцев кисти и распознает мимику на лице \cite{Holistic}.
  \item Halpe (см. \autoref{fig:Halpe_topology})\\
  Данная топология - это совместный проект AlphaPose \cite{fang2017rmpe} и HAKE \cite{li2020pastanet}. Представлено две модели: на 26 и на 136 точек. Здесь добавлено рассмотрение ориентации стоп, распознавание шеи, паха и макушки головы. В расширенной модели присутствует ещё 68 точек на лице, а также по 21 на ладонях.
\end{enumerate}

\begin{figure}[h]
\begin{subfigure}[b]{.3\textwidth}
	\centering
	\includegraphics[width=\textwidth]{./images/COCO_topology.jpg}
	\caption{Топология COCO}
	\label{fig:COCO_topology}
\end{subfigure}
\begin{subfigure}[b]{.3\textwidth}
	\centering
    \includegraphics[width=\textwidth]{./images/BlazePose_topology.jpg}
    \caption{Топология BlazePose }
    \label{fig:BlazePose_topology}
\end{subfigure}
\begin{subfigure}[b]{.3\textwidth}
	\centering
    \includegraphics[height=\textwidth]{./images/Halpe_topology.jpg}
    \caption{Топология Halpe}
    \label{fig:Halpe_topology}
\end{subfigure}
    \caption{Примеры расположения точек на теле человека.}
\end{figure}

В итоге мы разобрались с тем что нам необходимо искать в нашей работе и сейчас необходимо понять как это делать. Данная работа проводится в два этапа: 
\begin{enumerate}
	\item Локализация человека и его частей тела
	\item Упорядочивание и распределение суставов в правильном порядке.
\end{enumerate}

В реальном мире мы имеем два подхода к поиску ключевых точек и восстановлению скелета на изображении:
\begin{itemize}
	\item Bottom-up\\
	Когда сначала распознаем точки,  потом собираем их в скелет
	\item Top-down\\
	Сначала происходит локализация людей или объектов, а потом происходит распознование ключевых точек
\end{itemize}

ДУМАЮ ТУТ СТОИТ ПРОДОЛЖИТЬ ИЗ РАЗДЕЛА МНОГО ТЕОРИИ ДЛЯ РАЗБАВКИ РАБОТЫ. О ТОМ КАК ДЕТЕКТИРОВАТЬ И ИСКАТЬ ЭТИ ТОЧКИ. ТАКЖЕ МОЖНО ДОБАВИТЬ ПРО РАЗЛИЧНЫЕ ФУНКЦИИ И ФОРМУЛКИ. ДОЛЖНО ПОЛУЧИТСЯ КРАСИВО И ИНТЕРЕСНО.

\subsection{Задача классификации движений/позы человека}
\label{subsec:Theory of classification}

Исходя из информации о полученных ключевых точках можно сделать вывод о категории движения, которое может совершаться в данной позе.

Полученные координаты ключевых точек можно использовать как признаки для различных классификаторов. Думаю тут не стоит сильно заморачиваться с описанием задачи машинного обучения классификации. Может просто забить, а может расписать и добавить ещё несколько страниц. СПРОСИТЬ ПРО ДАННЫЙ РАЗДЕЛ!!!
\newpage