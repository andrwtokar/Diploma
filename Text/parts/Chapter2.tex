\section{Обзор существующих моделей}
\label{sec:Chapter2} \index{Chapter2}
Здесь надо рассмотреть все существующие решения поставленной задачи, но не просто пересказать, в чем там дело, а оценить степень их соответствия тем ограничениям, которые были сформулированы в постановке задачи.


\subsection{Модели для распознавания ключевых точек на теле человека}

В данном разделе мы рассмотрим 5-6 различных моделей. В главе \ref{sec:Chapter4} мы выберем 4 наиболее удобные в использовании и в обучении и проведем эксперимент по оценке данных моделей.

Так же хочется сказать, что, помимо приведенных, есть множество моделей от одиночных авторов, не объединенных в лаборатории (ССЫЛКИ). Они в основном брали какую-то из представленных моделей и проводили небольшое улучшение.

А теперь перейдем к моделям.

\subsubsection{BlazePose by MediaPipe}

MediaPipe является одним из проектов компании GOOGLE и в своей работе решает задачи компьютерного зрения. Там уже были модели для распзнавания лиц, ладоней и детекции человека на картинке. Для нас же интересна задача детекции ключевых точек, которую и решает моделй BlazePose. На данный момент модель может отслеживать движения на видеофрагменте, а также строить маску движения человка. При запуске тестового колаб ноутбука отработало все очень хорошо и приятно. Рассмотрим же дальше ее особенности.

\subsubsection{MoveNet.SinglePose.lightning by TensorFlowHub}

Данная модель представляет собой отрисовку персонально человечка. Есть более развитая версия этой модели, которая может в мультиперсональный трекинг.

\subsubsection{OpenPose by ...}

Азиаты

\subsubsection{MMPose by OpenMMlab}

Азиаты

\subsubsection{AlphaPose}

Азиаты?

\subsubsection{Detectron2 by FacebookResearch}

Работает не очень приятно, но как бы она есть и то ладно.


\subsection{Модели для классификации позы человека}

В данном разделе мы рассмотрим 4 различных моделей. Позже выберем 3 наиболее удобные в использовании и в обучении и проведем эксперимент по оценке данных моделей.

\subsubsection{MMaction2 by OpenMMlab}

Что-то сложное...

\subsubsection{BlazePose by MediaPipe}

Просто накинули сверху предыдущей модели kNN.

\subsubsection{mmakos}

Чувак взял OpenPose и добавил классификатор.

\subsubsection{HPC}

Уже и не помню про что тут...

\newpage