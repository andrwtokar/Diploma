\begin{abstract}

    \begin{center}
        \large{Исследование методов классификации движений человека} \\
    \large\textit{Токарев Андрей Сергеевич} \\[1 cm]
    \end{center}
    
	\begin{large}   
   После достижения результатов в задаче сегментации человека и отслеживании его на видео стало интересна возможность оценки занимаемой им позы и ее классификация. Для этого было необходимо научиться распознавать выбранные ключевые точки на теле человека, которые несут информацию о его положении. Некоторые группы исследователей уже добились хороших результатов в данном направлении и двигаются дальше в задачах анализа взаимодействия между людьми и предсказания следующих движений человека.
	\end{large}
  
	\begin{large}   
   В данной работе представлен обзор задачи классификации движений на основе информации о ключевых точках на теле человека, а также приведены технический обзор систем глубокого обучения для оценки позы и их исследование  с помощью публичных наборов данных.
	\end{large}

\end{abstract}
\newpage