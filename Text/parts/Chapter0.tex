\section{Введение}
\label{sec:Chapter0} \index{Chapter0}

В этой части надо описать предметную область, задачу из которой вы будете решать, объяснить её актуальность (почему надо что-то делать сейчас?).
Здесь же стоит ввести определения понятий, которые вам понадобятся в постановке задачи.

\hfill \break
\textbf{\Large 3 Вариант введения.}

Понимание движений человека является необходимой частью нашей жизни. При общении людьми часто используется жестикуляция, так как это помогает выражать чувства, эмоции и доносить свои мысли до окружающих . Из анализа позы человека можно сделать вывод о его состоянии. К примеру, хромота или нахождение в неестественном положении говорят о необходимости не только медицинской, но, возможно, и вашей помощи. Ещё можно обратится к психоанализу, а точнее к разделу о языке телодвижений. В нем по позе можно сделать вывод о характере человека или о текущем состоянии, его заинтересованности  в беседе.Также работает распознавание движений. Если мы видим бегущих в панике людей, то наш мозг получает сигнал об опасности и спасает нас. Из приведенных ситуацию   становится  понятно, почему определение позы и классификация движений являются важными аспектами нашей жизни. В связи с развитием информационных технологий, человечество задумалось над выполнением данной задачи с помощью компьютера. Тогда можно будет добавить дополнительный источник информации для  взаимодействия искусственного интеллекта с человеком.

При рассмотрении данной задачи через призму машинного обучения, получим, что нам нужно классифицировать положение человека, данные о котором необходимо каким-то образом получать. Первый способ - надеть на добровольца датчики и, считывая координаты каждого из них, построить на компьютере его позу и, таким образом, восстановить скелет для последующего анализа. Второй способ - искать особые точки на фотографии с помощью компьютерного зрения. Установим камеру и начнем анализировать положение и скелет человека, исходя из картинки. Тогда не придется закупать большое количество датчиков для снятия данных, а нужна будет только камера и вычислительные мощности для работы алгоритмов глубокого обучения. Несмотря на сложность реализации первого варианта, его удобно использовать для подготовки тренировочных датасетов.

\hfill \break
Движение - это растянутый во времени процесс. Он анализируется по видеозаписям, каждая из которых представляет собой последовательность кадров. Поэтому первостепенно научиться работать с изображением. Как же собирать данные для модели классификации?

Восстановление скелета (Skeletal Representation), детекция (Pose Detection) и оценка позы (Pose Estimation), распознавание движения (Action Recognition) являются расширением одной задачи: распознавание ключевых точек на теле человека (Key-points Detection). Задачи, которая имеет прикладной смысл не только в связке с классификацией движений. В работе мы будем рассматривать распознавание только на картинке, то есть в 2-х мерном пространстве. Но ведь можно восстанавливать положение человека (скелет человека) в 3-х мерном пространстве \cite{WANG2021103225} \cite{8100086}. Используя генеративные нейронные сети можно воссоздавать не только скелет человека, но и тело человека \cite{Zhang_2017_CVPR}. Объединяя две предыдущие задачи можно получить набор данных из 3-х мерных людей в различных позах. Некоторые исследователи уже пробуют реализовать этот симбиоз на практике \cite{varol17_surreal}.

Если перейти в тематику биологических и медицинских наук, то можно развить данную тему на примере восстановления структуры тканей человека. Получается, мы сможем по фотографии моделировать распределение мышечных, жировых и других тканей в теле человека. Это поможет более детально изучать проблемы персонально, каждого человека и подбирать индивидуальные курсы лечения или диеты.
Восстановление скелета человека поможет спасателям анализировать положение человека под завалами и строить планы по его спасению, имея более детальную информацию. Правда в данном случае необходимо быстродействие алгоритма и очень важно получить изображение человека.
В современном мире, где повсюду слышны разговоры о технологиях дополненной реальности и мета вселенной, найдем ещё одно применение для алгоритмов детектирования позы. Для нахождения в виртуальной вселенной необходимо транслировать человека туда, а значит можно с помощью видеокамер определять положение, восстанавливать скелет и получать итоговое изображение или 3-х мерную модель. Чем-то напоминает фильм "Первому Игроку Приготовиться"{} Стивена Спилберга. Добавим алгоритм генерации аватара вместо реального человека и получим рабочий алгоритм трансляции живого человека в мета вселенную.

\hfill \break
Второй частью работы является задача классификации, которая использует данные, полученные в первой части. Таким образом, мы построили алгоритм анализа движений человека на статическом изображении. По изображению мы не можем давать оценку поведению человека, но своеобразный "помощник"{} из полученного алгоритма будет хороший. Рассмотрим некоторые идеи применения.

Начать можно с медицины. Восстановление больных после операций, травм и несчастных случаев - это длительный и трудоемкий процесс, требующий постоянного присмотра врача. Если человек учится двигаться, то нужен тренер, который укажет  на ошибки и исправит вас. Решение нашей задачи помогает таким пациентам. Анализ движений может сравнивать человека с эталоном и указывать на ошибки. Также при наблюдении за больным алгоритм может идентифицировать отклонения от нормального поведения и вызвать врача (к примеру увидеть приступы эпилепсии у человека). Это может спасти множество жизней по всему миру, просто вызывая врача в необходимый момент, а также помочь в востановлении.

Также можно выявлять у здорового человека заболевания или дефекты скелета. Можно анализировать сколиоз или сутулость и подсказывать людям, что надо стараться держать спину прямо. Хотя лучше направлять к врачу на консультацию и лечить дефекты позвоночника сразу. Проведя исследование населения, мы получим статистику тех или иных отклонений. Так уже сделали производители кроссовок и с помощью gait-анализа \cite{WHITTLE1996369} помогают выбрать подходящую обувь.

Посмотрим теперь на спорт. Из классификатора можно сделать хорошего судью соревнований в тех видах, где надо различать, отслеживать положения тела. К примеру, GOOGLE придумали использовать классификатор как счетчик подтягиваний, приседаний или отжиманий \cite{counter} и это можно поместить в современный смартфон. Если углубиться дальше, то решение можно обернуть интерфейсом и создать хорошего робота-фитнес-тренера. Ведь настроив камеру смартфона на наблюдение за вами во время тренировки, приложение будет подсказывать вам правильную позу для упражнения и укажет на ошибки, если таковые имеются.

\hfill \break
При развитии моделей в будущем, можно будет найти другие варианты применения технологии классификации движений человека. Можно анализировать поведение группы людей, но для этого надо хорошо восстанавливать скелет нескольких человек на одном изображении \cite{8765346} \cite{https://doi.org/10.48550/arxiv.1807.04067} \cite{fang2017rmpe}. Также есть возможность предсказывать будущие действия человека при изучении уже имеющихся. Если опять затронуть идею генеративных нейронных сетей, то можно генерировать движения человека по заданному начальному условию. Следовательно, можно создавать искусственные видеозаписи или добавлять неигровых персонажей (npc - non-player character) в виртуальную реальность. В медицине можно моделировать восстановление двигательной активности человека или моделирование протезов индивидуально под каждого пациента, чтобы он мог восстановить большую часть двигательной активности.

Как можно заметить, применений можно придумать множество - необходимо реализовать проект и получить модель. В текущий момент в мире уже существует какое-то количество решений описанных выше задач. В предложенной вашему чтению работе я рассмотрю некоторые из них, приведу качественную оценку результатам эксперимента и сделаю вывод с определением дальнейшего моего развития в данной теме. (МОЖЕТ СТОИТ УБРАТЬ ПРО МОЕ РАЗВИТИЕ В ДАННОЙ ТЕМЕ?)



\hfill \break
\textbf{\Large 2 Вариант введения. Вроде не законченный.}

После достижение определенных хороших результатов в области детекции объектов на изображениях необходимо было двигаться дальше. Одним из дальнейших направлений стала классификация движений человека с помощью нейронных сетей. Пусть на данный момент уже делаются попытки предсказывать будущие движения человека по предыдущим кадрам видеозаписи (НЕОБХОДИМА ССЫЛКА), но в данной работе мы сфокусируемся на анализе современных методов классификации движений человека. Так как данная задача подразделяется на две части: распознавание ключевых точек на теле человека и классификация движения человека исходя из анализа результатов прошлой части.

Первая часть работы имеет дальнейшее развитие, не связанное с классификацией движений,  в задачу восстановления скелета человека в 3-х мерном пространстве (ССЫЛКА) и в задачу о генерации данных (ССЫЛКА): синтетические люди в 3-х мерном пространстве. Эти задачи могут помочь как ученым в моделировании человеческих поз и их дальнейшем изучении, так и комерческим компаниям в производстве акксессуаров дополненной реальности. А так как в последнее время многие компании развивают идеи виртуальной реальности, то необходимо каким-то образом подгружать человека в виртуальный мир. Одним из способов может стать съемка его через камеру. Представьте через несколько лет виртуальную конференцию в большой компании, где все участники находятся в виртуальном мире через веб-камеру. Интересно, не так ли?

Вторая же часть работы является развитием первой части. Исходя из состояния ключевых точек мы можем построить 2-х мерный скелет, по которому можно проанализировать позу человека. Анализируя позу мы можем сделать вывод о том, какое движение человек выполняет. В этом и состоит анализ движения человека. Развивается данная идея в анализ видео последовательности кадров, в анализ межчеловеческого взаимодействия на картинках и в предсказание будущих движений человека по уже имеющимся.  

\hfill \break
\textbf{\Large 1 Вариант введения.}

В данной работе будет представлен анализ современных систем распознавания ключевых точек на теле человека и классификации движений человека по данным полученным данным. На сегодняшний момент мир довольно сильно развился и требует анализа человеческих действий с помощью искуственного интеллекта. Исходя их таких запросов был сформирован план действий по достижению таких результатов. Такие моменты как переход от 2D картинки  к 3D можелированию скелета человека, сложные позы и разный рост людей решаются в данный момент различными способами, которые будут описаны ниже.

Области применения этой задачи могут быть весьма разнообразные: виртуальная реальность (VR, AR), медицина (восстановлене больных и слежение за их движениями - правильно или неправильно они двигаются), фитнес-приложения (отслеживать правильность выполнения упражнения), производство различных аксессуаров (если человек работает в определенной позе, то можно смоделировать его положение и понять каким образом лучше реализовать инструмент или аксессуар). Одна компания анализировала постановку стопы человека (необходимо добавить ссылку на информацию), чтобы принять решение о выпуске будущей коллекции обуви. Оказалось, что большинство людей заваливают ногу внутрь и из-за этого им лучше покупать специальную обувь. Также можно анализировать правильность выпонения упражнения в спорте, а также вести подсчет повторений на соревнованиях (пожтягивание, отжимание или приседание), ведь если человек не примет верную позу, то повторение не будет засчитано :)

По моим ощущениям данная задача является одной из тех задач, которые помогут упростить жизнь на нашей планете и сделать ее более безопастной и продвинутой. Данная технология поможет во многих отраслях жизни, поэтому я бы хотел продолжить развиваться в области данной задачи.


\newpage