\section{Введение}
\label{sec:Chapter0} \index{Chapter0}

В этой части надо описать предметную область, задачу из которой вы будете решать, объяснить её актуальность (почему надо что-то делать сейчас?).
Здесь же стоит ввести определения понятий, которые вам понадобятся в постановке задачи.

\hfill \break
\textbf{\Large 3 Вариант введения.}

Понимание движений человека является необходимой частью нашей жизни. При общении с людьми мы всегда используем жестикуляцию (МОЖЕТ СТОИТ ПРИСМОТРЕТЬ ДРУГОЕ СЛОВО???), так как она помогает нам выражать чувства, эмоции или доносить свои мысли до других. Если человек хромает или находится в неестественной позе, то, возможно, он болен и ему необходимо помочь. Если люди бегут от какой-то опасности, то мы тоже побежим с ними и этим спасем свою жизнь. Хотя мы можем даже не видеть эту опасность. В то же время сейчас есть много литературы по поводу анализа человека по позам, которые он принимает (НЕОБХОДИМЫ ССЫЛКИ). Психоаналитики и биологи продвинулись довольно далеко в этом плане. Думаю стало понятнее, почему определение позы и классификация движений являются довольно важными аспектами нашей жизни. Поэтому человечество задумалось над тем, чтобы классифицировать движения человека с помощью компьютера. Это будет полезно для коммуникации с искусственным интеллектом (ИИ) к которому человечетсво стремится.

Если ставить данную задачу в области машинного обучения, то мы будем классифицировать движения человека исходя из данных о его позе (СТОИТ ПОДУМАТЬ НАД ПЕРЕФРАЗИРОВАНИЕМ). Можно навесить на человека датчиков, узнавать в каком положении находится каждый датчик и таким образом строить скелет человека для последующего анализа. В то же время, мы имеем довольно перспективную отрасль машинного обучения, как компьютерное зрение. Попробуем установить камеру и анализировать позу и скелет человека исходя из картинки. Тогда не надо будет закупать множество датчиков для снятия данных. Нужна будет только камера и некоторые вычислительные мощности для работы алгоритмов глубокого обучения. В реальности получилось так, что в повседневной жизни второй вариант проще, но пр помощи первого варианта получения данных удобно получать точные данные для обучения моделей. Поэтому остановимся на развитии компьютерного зрения.

Движение человека - длительный во времени процесс и, чтобы анализировать его, мы должны рассматривать видеозаписи. Но каждая видеозапись - это отдельная фотография. Поэтому начнем с того, чтобы классифицировать движение человека по отдельной фотографии. Как уже было сказано выше, для классификации нам необходимы данные о позе человека.

\hfill \break
Детектирование позы человека может иметь большой прикладной смысл не только в связке с задачей классификации движений. В работе мы будем рассматривать только распознавание скелета на картинке, то есть в 2-х мерном пространстве. Но ведь можно восстанавливать положение человека (далее скелет человека) в 3-х мерном пространстве (ССЫЛКА). Используя генеративные нейронные сети можно создавать новые датасеты (НАБОРЫ ДАННЫХ) для данных задач - таким образом замыкая цикл машинного обучения (ПРО ЦИКЛ МОЖЕТ СТОИТ УБРАТЬ). Объединяя две предыдущие задачи можно получить набор данных из 3-х мерных людей в различных позах. Некоторые исследователи уже пробуют реализовать этот симбиоз на практике (ССЫЛКА). 

При использовании данной задачи в биологических и медицинских целях можно развивать данную задачу для восстановления мышц и тканей человека. Это поможет исследовать возможноти человеческого организма в той или иной позе. ДОБАВИТЬ...

В современном мире, где повсюду можно услышать множество разговоров про виртуальный мир, создание приложений и аксессуаров дополненной реальности и метавселенную (ССЫЛКИ), можно увидеть ещё одно применение для алгоритмов детектирования позы. Для нахождения в виртуальной вселенной необходимо транслировать человека туда, а значит можно с помощью видеокамер определять его положение, восстанавливать скелет и получать итоговое изображение или 3-х мерную модель. Чем-то напоминает фильм "Первому Игроку Приготовиться"{} Стивена Спилберга. Добавим алгоритм генерации аватара вместо реального человека и получим рабочий алгоритм трансляции живого человека в виртуальную вселенную.

\hfill \break
Вернемся к классификатору. Данные мы получили, добавили хороший алгоритм классификации и решили задачу. Помимо помощи при общении и при взаимодействии, можно найти ещё несколько применений данной работы. 

Начать можно с медицины. Восстановление больных после операций, травм и несчатных случаев - это длительный и трудоемкий процесс, требующий постоянного присмотра врача. Если человек учится двигаться, то нужен тренер, который покажет на ошибки и исправит вас. Решение нашей задачи помогает таким пациентам. Анализ движений может сравнивать человека с эталоном и указывать на ошибки. Также при наблюдении за больным алгоритм может идентифицировать отклонения от нормального поведения и вызвать врача (к примеру увидеть приступы эпилепсии у человека). Это может спасти множество жизней по всему миру, просто вызывая врача в необхоодимый момент, а также помочь в востановлении.

Также можно выявлять у здорового человека заболевания или дефекты скелета. Можно анализировать скалиоз или сутулость и подсказывать людям, что надо стараться держать спину прямо. Хотя лучше направлять к врачу на консультацию и лечить дефекты позвоничника сразу. Проведя исследование населения мы получим статистику наличия у населения тех или иных отклонений. Так уже сделали производители кроссовок (ССЫЛКА) и узнали, что при хотьбе или беге у большинства людей стопа заваливается внутрь. Этот анализ помог сделать ориентацию производства на покупателя для сохранения его здоровья.

Дописать про спорт...




В предложенной работе я рассмотрю некоторые из уже существующих моделей для анализа движений человека по изображениям.

\hfill \break
\textbf{\Large 2 Вариант введения. Вроде не законченный.}

После достижение определенных хороших результатов в области детекции объектов на изображениях необходимо было двигаться дальше. Одним из дальнейших направлений стала классификация движений человека с помощью нейронных сетей. Пусть на данный момент уже делаются попытки предсказывать будущие движения человека по предыдущим кадрам видеозаписи (НЕОБХОДИМА ССЫЛКА), но в данной работе мы сфокусируемся на анализе современных методов классификации движений человека. Так как данная задача подразделяется на две части: распознавание ключевых точек на теле человека и классификация движения человека исходя из анализа результатов прошлой части.

Первая часть работы имеет дальнейшее развитие, не связанное с классификацией движений,  в задачу восстановления скелета человека в 3-х мерном пространстве (ССЫЛКА) и в задачу о генерации данных (ССЫЛКА): синтетические люди в 3-х мерном пространстве. Эти задачи могут помочь как ученым в моделировании человеческих поз и их дальнейшем изучении, так и комерческим компаниям в производстве акксессуаров дополненной реальности. А так как в последнее время многие компании развивают идеи виртуальной реальности, то необходимо каким-то образом подгружать человека в виртуальный мир. Одним из способов может стать съемка его через камеру. Представьте через несколько лет виртуальную конференцию в большой компании, где все участники находятся в виртуальном мире через веб-камеру. Интересно, не так ли?

Вторая же часть работы является развитием первой части. Исходя из состояния ключевых точек мы можем построить 2-х мерный скелет, по которому можно проанализировать позу человека. Анализируя позу мы можем сделать вывод о том, какое движение человек выполняет. В этом и состоит анализ движения человека. Развивается данная идея в анализ видео последовательности кадров, в анализ межчеловеческого взаимодействия на картинках и в предсказание будущих движений человека по уже имеющимся.  

\hfill \break
\textbf{\Large 1 Вариант введения.}

В данной работе будет представлен анализ современных систем распознавания ключевых точек на теле человека и классификации движений человека по данным полученным данным. На сегодняшний момент мир довольно сильно развился и требует анализа человеческих действий с помощью искуственного интеллекта. Исходя их таких запросов был сформирован план действий по достижению таких результатов. Такие моменты как переход от 2D картинки  к 3D можелированию скелета человека, сложные позы и разный рост людей решаются в данный момент различными способами, которые будут описаны ниже.

Области применения этой задачи могут быть весьма разнообразные: виртуальная реальность (VR, AR), медицина (восстановлене больных и слежение за их движениями - правильно или неправильно они двигаются), фитнес-приложения (отслеживать правильность выполнения упражнения), производство различных аксессуаров (если человек работает в определенной позе, то можно смоделировать его положение и понять каким образом лучше реализовать инструмент или аксессуар). Одна компания анализировала постановку стопы человека (необходимо добавить ссылку на информацию), чтобы принять решение о выпуске будущей коллекции обуви. Оказалось, что большинство людей заваливают ногу внутрь и из-за этого им лучше покупать специальную обувь. Также можно анализировать правильность выпонения упражнения в спорте, а также вести подсчет повторений на соревнованиях (пожтягивание, отжимание или приседание), ведь если человек не примет верную позу, то повторение не будет засчитано :)

По моим ощущениям данная задача является одной из тех задач, которые помогут упростить жизнь на нашей планете и сделать ее более безопастной и продвинутой. Данная технология поможет во многих отраслях жизни, поэтому я бы хотел продолжить развиваться в области данной задачи.


\newpage