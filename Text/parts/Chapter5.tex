\section{Заключение}
\label{sec:Chapter5} \index{Chapter5}

В ходе работы было рассмотрено несколько решений задачи классификации движений человека, в основе которых лежат модели распознавания  ключевых точек. Так как данная задача выглядит более простой в сравнении с задачей получения векторов признаков - были рассмотрены современные проекты для оценки позы.

Результаты эксперимента показали:
\begin{itemize}
	\item При визуальной оценке работы различных моделей (см. \autoref{sec:Appendix}) выделяется OpenPose как адекватностью локализации, так и качеством отрисовки изображения;
	\item Модель MoveNet показывает самое быстрое распознавание точек, при самой низкой точности распознавания;
	\item BlazePose является лучшим решением из представленных по точности детектирования ключевых точек.
\end{itemize} 

Подводя итог, 3 модели из 4 показывают примерно одинаковый результат при разных затратах времени на работу. MMPose, хоть и показывает высокие показатели качества, имеет очень большое время обработки изображения, которое почти в 10 раз превышает другие модели. Если допустима высокая погрешность при очень быстрой работе, то стоит посмотреть на модель MoveNet. При пороге 0.5 на PCK и PDJ она не сильно отстает от других участников эксперимента.

Дальнейшая цель исследования заключается в том, чтобы рассмотреть задачу анализа движений человека по видео и попробовать улучшить некоторые модели посредством архитектурных изменений или используя обучение на узко ориентированном наборе данных.





\newpage