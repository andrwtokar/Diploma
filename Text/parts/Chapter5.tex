\section{Заключение}
\label{sec:Chapter5} \index{Chapter5}

В ходе работы было рассмотрено несколько решений задачи классификации движений человека, в основе которых лежат модели распознавания  ключевых точек. Так как данная задача выглядит более простой в сравнении с задачей получения векторов признаков - были рассмотрены современные проекты для оценки позы.
\hspace{1cm}

Результаты эксперимента показали:
\begin{itemize}
	\item При визуальной оценке работы различных моделей (см. \autoref{sec:Appendix}) выделяются OpenPose и MoveNet как адекватностью локализации, так и качеством отрисовки изображения;
	\item Модель MoveNet показывает самое быстрое распознавание точек, правда точность модели уступает предыдущим двум;
	\item Модели OpenPose и MMPose являются лучшими решениями из представленных по точности детектирования ключевых точек.
\end{itemize} 

Подводя итог, если не обращать внимание на высокую точность (порог метрик PDJ и PCK 0.2 и больше) можно выбрать модель MoveNet - так как она показала себя хорошо в данном вопросе. Если необходима достоверность рассматриваемых результатов, то стоит обратиться к OpenPose или MMPose. Но стоит отметить, что первая работает значительно быстрее, чем вторая.

Дальнейшая цель исследования заключается в том, чтобы рассмотреть задачу анализа движений человека по видеофрагментам и попробовать улучшить некоторые модели посредством архитектурных изменений или используя обучение на узко ориентированном наборе данных.





\newpage