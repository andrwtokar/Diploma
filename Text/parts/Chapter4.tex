\section{Исследование моделей}
\label{sec:Chapter4} \index{Chapter4}

В данной главе я опишу поставленный эксперимент по исследованию моделей и приведу полученные рузльтаты.

\subsection{Описание эксперимента}

Эксперимент, как и вся работа разделен на две части: исследование моделей распознавания ключевых точек на теле человека и исследование моделей классификации поз человека.

В первой части мы рассмотрели 4 модели на работоспособность. Все они показали хороший результат классификации. Для определения наиболее хорошей можели использовались метрики: РСК и PDJ (возможно OKS, но с ней пока что я не разобрался). Метрика высчитывалась на точках туловища, так как в датасете нет размеченных точек лица (только верхушка головы, а она в исследуемых моделях не присутствует).

Во второй части ... ее ещё надо написать и создать. Надеюсь успеть это сделать на выходных...

\subsection{Полученные результаты эксперимента}

Выбранный мной датасет имеет фотографии низкого качества и маленького размера, поэтому приведу показ работоспособности моделей на фотографиях собственной работы. 

(ФОТОЧКИ)

Также хочу привести результаты высчитывания метрик для нескольких моделей и различных пороговых значений в метриках.

В дополнение приведем средние значения по обработке одного изображения моделью.

Вторую чать работы пишем...

\newpage