\section{Исследование моделей}
\label{sec:Chapter4} \index{Chapter4}

В данной главе я опишу поставленный эксперимент по исследованию моделей и приведу полученные рузльтаты.

\subsection{Описание эксперимента}

Эксперимент, как и вся работа разделен на две части: исследование моделей распознавания ключевых точек на теле человека и исследование моделей классификации поз человека.

В первой части работы были выбраны несколько моделей распознавания ключевых точек на теле человека и проведен их анализ. Для выбора моделей использовались некоторые критерии:

\begin{itemize}
	\item Доступность модели для исследований\\
	Необходимо оценить длительность установки и возможности работы с различными операционными системами. Эксперимент проводился на платформе Google Colab, поэтому необходимо было рассмотреть возможность использования модели на в Colaboratory.
	\item Новизна модели\\
	Представленная выборка была создана в основном в 2010-х, но модель DeepPose является самой старой. Новые разработки опирались на результаты, полученные в ней, и таким образом получали более хорошие результаты.
	\item Наличие документации\\
	Все модели производят классификацию по двум осям изображения: выстоа и ширина, а также по параметру видимость ключевой точки. Некоторые модели выдают данные нормированные на размер изображения (число из отрезка [0,1]), а некоторые точное значение в пикселях. Поэтому для работы необходимо было понимать как работает API модели, какие у нее входные - выходные данные.
	\item Тренировка модели на датасете COCO (ССЫЛКА)\\
	Все используемые претренированные модели были обучены на наборе данных COCO в совместительстве с каким-либо другим датасетом. В некоторых  примерах не было возможности использовать претренированную модель и из-за этого они были отсеяны
\end{itemize}
В итоге было выбрано 4 модели наиболее подходящие под критерии:
\begin{enumerate}
	\item BlazePose
	\item MoveNet.SinglePose
	\item OpenPose
	\item MMpose
\end{enumerate}

Дальнейшим пунктом был анализ работы выбранных детекторов по некоторым метрикам:

\begin{itemize}
	\item PCK (Percentage of Correct Key-points)\\
	Метрика оценивает точность распознавания ключевой точки в зависимости от диагональных размеров человека. Поэтому необходимо было получать диагональ рамки детекции человека, как объекта, и использовать ее в высчитывании метрики. (ФОРМУЛА)\\
	Порогом варьируется допустимая погрешность расстояния между реальной и предсказанной точками. (КАРТИНКА)
	\item PDJ (Percentage of Detection Joints)\\
	PDJ очень похожа на PCK, только за размер человека берется не диагональ его тела, а высота его тела. (ФОРМУЛА)
	\item OKS (Object Key-point Similarity)\\
	Данная метрика является основной при оценке задачи Keypoint Detection COCO (ССЫЛКА). Она использует третью координату выходного предсказания и расстояние между реальной и предсказанной точками. (ФОРМУЛА)\\
	При обособлении данной метрики можем получить метрику mAP, которая является просто OKS с различными порога считывания.	
\end{itemize}

Во второй части ... ее ещё надо написать и создать. Надеюсь успеть это сделать на выходных...

\subsubsection{Поиск данных}

Первым делом необходимо было проверить модели распознавания ключевых точек на внешнюю адекватность работы на неразмеченных данных. Для этого были выбраны некоторые фотографии высокого разрешения, где человек полностью присутствует на фотографии. (ФОТОГРАФИИ)

Далее необходимо было найти размеченные данные для оценки детекторов с помощью метрик. Приведу описание найденных размеченных датасетов:

\begin{itemize}
	\item COCO Dataset (ССЫЛКА)\\
	Набор данных является основным в задачах распозавания ключевых точек. Все, используемые в эсперименте модели, были претренированы на нем. Поэтому для оценки работы его использовать не является целесообразным.
	\item MPII (ССЫЛКА)\\
	Большая база данных для различных задач компьютерного зрения. При его рассмотрении были замечены моменты, что человек не полностью виден на фотографии,а также модели (ПРИВЕСТИ ТОЧНЫЕ НАЗВАНИЯ) используют его как тренировочных в добавку к датасету COCO.
	\item HUMAN 3.6M (ССЫЛКА)\\
	Данных набор данных идеально подошел был для работы, но доступ к нему ограничен и создатель набора не выходит на связь.
	\item SURREAL (ССЫЛКА)
	Было интересно поработать со сгенерированным датасетом, но он тоже с закрытым доступом. Для работы в магистратуре постараюсь добиться доступа к нему и провести исследования.
	\item LSP (ССЫЛКА)\\
	Позы в спорте. Его используем. Есть одна проблема с оценкой распознавания точек на лице - в датасете они не описаны. Поэтому смотрим только классификацию суставов или точек на теле, другими словами.
\end{itemize}

Выше были представлены наборы данных для задачи распознавания точек на теле челловека. Но основной темой является классификация движений человека. Поэтому необходимы фотографии с меткой класса для позы, представленной на данных. Некоторые из уже представленных (COCO, MPII) тоже могут использоваться для классификации позы, но по тем же причинам, что и описаны выше, они не будут рассмотрены в эксперименте. Приведу примеры наборов данных для классификации движения человека по позе на изображении:

\begin{itemize}
	\item HPC/mmakos (ССЫЛКА)\\
	\item Stanford-40 (ССЫЛКА)\\
	\item Yoga-82 (ССЫЛКА)\\
\end{itemize}

Итого были выбраны наборы данных: LSP и ... (ФОТОГРАФИИ ИЗ ДАТАСЕТА)

\subsection{Полученные результаты эксперимента}

Рассмотрим результаты первой части работы.

Для каждой модели была рссматрена локализация ключевых точек на фотографиях высокого качества (ССЫЛКА НА ФОТО ВЫШЕ), а также набор данных низкого качества с обработанными изображениями. 

Метрики были высчитаны с порогами 0.05, 0.2, 0.5. В дополнение к этому был проведен временной анализ классификации одного изображения в среднем по целому датасету.

Перейдем к результатам по каждой модели.

\begin{large}
BlazePose

ФОТОГРАФИИ МОИ И РЕЗУЛЬТАТЫ В ТАБЛИЦЕ

MoveNet.SinglePose

ФОТОГРАФИИ МОИ И РЕЗУЛЬТАТЫ В ТАБЛИЦЕ

OpenPose

ФОТОГРАФИИ МОИ И РЕЗУЛЬТАТЫ В ТАБЛИЦЕ

MMPose

ФОТОГРАФИИ МОИ И РЕЗУЛЬТАТЫ В ТАБЛИЦЕ
\end{large}

Вторую чать работы пишем...

\newpage